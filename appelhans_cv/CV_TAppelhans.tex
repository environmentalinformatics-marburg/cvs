\documentclass[a4paper,11pt]{article}

\usepackage{cv}
\usepackage{eurosym}
\usepackage{hyperref}
\usepackage{multirow}

\name{Tim Appelhans}
\info{Address: & Environmental Informatics, Department of Geography, Philipps University of Marburg, Deutschhausstr. 12, 35032 Marburg, Germany\\
      Phone: & +49 (0) 6421 28-25957\\
      Email: & tim.appelhans@staff.uni-marburg.de\\
      WWW: & http://umweltinformatik-marburg.de/mitarbeiterinnen-und-mitarbeiter/tim-appelhans/}

\addtocategory{papers}{Zawar-Reza2010, Appelhans2010a, Appelhans2010b, Appelhans2013, Kuehnlein2013,
                       Appelhans2015b, Appelhans2015, Appelhans2015c, Kuehnlein2014a, Kuehnlein2014, Appelhans2016,
                       Classen2015, Meyer2016}
\addtocategory{techreports}{Sturman2006, Appelhans2007, Wilton2009}
\addtocategory{conferences}{Appelhans2008, Appelhans2009, Appelhans2011, Appelhans2013a, Appelhansegu2015a, Appelhansegu2015b}
\addtocategory{curpapers}{mapview, satellite, remote}

\bibliography{appelhans}

%\makebibcategory{posters}{Conference posters}

\begin{document}

\maketitle
\section{Education and Qualifications}
\hrulefill \\
\begin{tabular}{lll}\\
since 08/2012 & \textbf{Lecturer (Akad. Rat)} & Philipps University Marburg\\
2011 - 2012 & \textbf{Post-Doctoral Fellow} & Philipps University Marburg\\
2010 - 2011 & \textbf{Post-Doctoral Fellow} & University Bayreuth\\
2010 & \textbf{Ph.D. Geography} & University of Canterbury\\
2008 & \textbf{Lecturer} & University of Otago\\
2005 & \textbf{Dipl. Geography} & Friedrich-Alexander University Erlangen-N\"urnberg
\end{tabular}

\section{Awards}
\hrulefill \\
\begin{tabular}{lll}\\
2010 & \textbf{Best Doctoral Thesis in Geography}. Presidents award, New Zealand Geographical Society.\\
2006 & \textbf{Best oral student presentation}. Resource Management Under Stormy Skies Conference, \\
	 & Christchurch, New Zealand, 20 - 23 November 2006.
\end{tabular}

\section{Research}
\hrulefill \\
\hfill \\
My principle research interests lie in the fields of geography, atmospheric sciences and ecosystem research across a wide range of spatial and temporal scales. In particular I am interested in boundary layer climatology and its interaction with other aspects of the earth-atmosphere system, especially in complex terrain (primarily montane and urban environments). My research is application-oriented and, being a geographer, I approach it in an inter-disciplinary manner. Among other research tools, I primarily use computational statistics (R), remote sensing and general spatial analysis approaches and, to a lesser extent, numerical modelling (WRF, TAPM) for my scientific investigations.\\
To date, I have authored \ref{sumpapers} papers, technical reports, conference contributions and software packages on various topics in the broad areas of environmental sciences, general geography and (applied) climatology. A list of these appears on pages \pageref{papersstart}--\pageref{papersend}.

\section{Grants}
\hrulefill \\
\hfill \\
\begin{tabular}{lll}
\textbf{2013 - 2016} & \textit{Ecological Climatology and Remote Sensing} & \textbf{\EUR{145,600}}\\
\end{tabular}
\linebreak{}
\linebreak{}
Together with Prof. Dr. Thomas Nauss from Philipps University Marburg I am leading this subproject which is part of the DFG research group FOR 1246 \textit{Kilimanjaro ecosystems under global change: Linking biodiversity, biotic interactions and biogeochemical ecosystem processes.}


\pagebreak{}

\section{Teaching}
\hrulefill \\
\begin{tabular}{ll} \\
	   \begin{minipage}[t]{0.2\textwidth}
       \raisebox{16mm} {\textbf{Lectures}}
       \end{minipage}
        & \parbox{0.75\textwidth}{
            \begin{itemize} \itemsep1pt \parskip0pt \parsep0pt
               	\item Climatology, GEOG 286/392, Otago (S1 2008)
                \item Environmental hazards management, GEOG 305, Canterbury (S1 2009, S1 2010)
                \item Environmental Processes: Research Practice, GEOG 211, Canterbury (S1 2010)
				\item Research Methods in Geography, 309, Canterbury (S2 2010)
				\item Recourses and Sustainability, 108, Canterbury (S2 2010)
            \end{itemize} }
\\      \begin{minipage}[t]{0.2\textwidth}
       \raisebox{15mm} {\textbf{Seminars}}
       \end{minipage}
        & \parbox{0.75\textwidth}{
            \begin{itemize} \itemsep1pt \parskip0pt \parsep0pt
                \item Analyse und Visualisierung von Umweltdatens\"atzen f\"ur den Einsatz in Beruf und Schule, Marburg (SS 2013) (Analysis and visualisation of enviromental data sets for professional use)
                \item Erfassung, Analyse und Visualsierung ausgew\"ahlter Umweltdatens\"atze, Marburg (WS 2012) (Collection, analysis and visualisation of selected environmental data sets)\\
                \item Aufbereitung, Analyse und Visualisierung von klima-\"okologischen Datens\"atzen, Marburg (WS 2011) (Handling, analysis and visualisation of eco-climatological data sets)\\
				\item Projektarbeit Physische Geographie, Marburg (WS 2011) (Project work physical Geography)
            \end{itemize} }
\\      \begin{minipage}[t]{0.2\textwidth}
       \raisebox{0mm} {\textbf{Laboratory courses}}
       \end{minipage}
        & \parbox{0.75\textwidth}{
            \begin{itemize} \itemsep1pt \parskip0pt \parsep0pt
                \item Climatology, GEOG 286/392, Otago (S1 2008)
            \end{itemize} }
\\      \begin{minipage}[t]{0.2\textwidth}
       \raisebox{2mm} {\textbf{Excursions/Practicals}}
       \end{minipage}
        & \parbox{0.75\textwidth}{
            \begin{itemize} \itemsep1pt \parskip0pt \parsep0pt
                \item Field research methods (Science), GEOG380, Otago (S1 2008)
                \item 4-t\"agige Exkursion Berchtesgaden, Bayreuth (SS 2011)
            \end{itemize} }
\end{tabular}

\pagebreak{}
\section{Grad student supervision}
\hrulefill \\
\begin{tabular}{llll} \\
\textbf{Ph.D.} & ongoing & I. Otte & Development of a new approach for cost-effective, ground-\\
	  & & & based fog remote sensing techniques at Mt. Kilimanjaro\\
	  & ongoing & F. Detsch & Quantification of evapo-transpiration in tropical ecosystems:\\
	  & & & an integrative approach using field observations and\\
	  & & & remote sensing techniques\\
	  & ongoing & E. Mwangomo & Classical spatial statistics vs. modern machine learning\\
	  & & & approaches for the generation of high-resolution\\
	  & & & climatological surfaces in complex terrain (Mt. Kilimanjaro)\\
	  & ongoing & H. Meyer & High resolution satellite- and machine learning based\\
      & & & monitoring of climate and land cover dynamics in\\
      & & & South African savannas\\
      & completed & M. Kuehnlein & A machine learning based 24-h-technique for an\\
      & & & area-wide rainfall retrieval using MSG SEVIRI data over\\
      & & & Central Europe\\
\end{tabular}

\section{Tertiary education training}
\hrulefill \\
\begin{tabular}{llll} \\
\textbf{Fortbildungszentrum Hochschullehre} & Planung einer Lehrveranstaltung\\
											& (structured course palnning) (12 AE/hrs)\\
\textbf{Hochschuldidaktisches Netzwerk Mittelhessen} & Fachliche und \"uberfachliche Kompetenzen\\
			& st\"arken durch reflektierte Projektarbeit in\\
			& gemeinn\"utzigen Kontexten:\\
			& das Service Learning Konzept\\
			& (project work and service learning) (16 AE/hrs)
\end{tabular}

\pagebreak{}
\section{Skills}
\hrulefill \\
\begin{tabular}{llll} \\
\textbf{Advanced knowledge} & of statistical programming including \\
							& data mining and machine learning applications (R)\\
\textbf{Advanced knowledge} & of Geographical Information Systems\\
				   & (R, IDRISI, QGIS, SAGA GIS, ESRI, GDAL, TNTmips)\\
				   & and other spatial/atmospheric analysis tools\\
				   & (incl. Surfer, IDV, Vapor)\\
\textbf{Proficient knowledge} & of UNIX/LINUX shell environment\\
\textbf{Basic knowledge} & of meso-scale numerical modelling\\
				   & (The Air Pollution Model - TAPM, WRF)\\
				   & and programming languages C++, javascript\\
\end{tabular}
\linebreak{}

\section{Software}
\hrulefill \\
\hfill \\
Since 2011 I have authored and contributed to various open source software programs/packages. Details below.
\linebreak{}
\linebreak{}
\begin{tabular}{ll}
\textbf{julendat} & JULENDAT Utilities for Environmental Data\\
				  & \url{https://github.com/environmentalinformatics-marburg/julendat}\\
\textbf{remote} 	  & Empirical Orthogonal Teleconnections in R\\
				  & \url{https://cran.r-project.org/web/packages/remote/index.html}\\
\textbf{satellite} & Various Functions for Handling and Manipulating Remote Sensing Data\\
				  & \url{https://cran.r-project.org/web/packages/satellite/index.html}\\
\textbf{mapview} & Interactive viewing of spatial objects in R\\
				  & \url{https://cran.r-project.org/web/packages/mapview/index.html}\\
\textbf{Rsenal}  & magic R functions for things various\\
				 & \url{https://github.com/environmentalinformatics-marburg/Rsenal}\\
\end{tabular}
\linebreak{}

\section{Administrative experience}
\hrulefill \\
\begin{tabular}{llll} \\
\textbf{since 2015} & Member of the Marburg Research Academy board of directors\\
\textbf{2010} & Administration of all laboratory courses at 100 level in Geography,\\
			  & Department of Geography, University of Canterbury.\\
\textbf{2007 - 2008} & PhD representative. Department of Geography,\\
			  & University of Canterbury, Christchurch, New Zealand.
\end{tabular}
\linebreak{}
\pagebreak{}

\section{References}
\hrulefill \\
\begin{tabular}{llll} \\
\textbf{Prof. Andrew Sturman} & Department of Geography, University of Canterbury,\\
					 & Private Bag 4800, Christchurch, New Zealand.\\
					 & email: andrew.sturman@canterbury.ac.nz\\
					 & phone: +64 3 364 2502\\

\textbf{Prof. Dr. Thomas Nauss} & Environmental Informatics, Department of Geography,\\
					   & Philipps University Marburg, Deutschhausstr. 12,\\
					   & 35032 Marburg, Germany.\\
					   & email: thomas.nauss@staff.uni-marburg.de\\
					   & phone: +49 6421 28 25980\\

\textbf{Dr. Nicolas Cullen} & Department of Geography, University of Otago,\\
				   & PO Box 56, Dunedin, New Zealand.\\
				   & email: njc@geography.otago.ac.nz\\
				   & phone: +64 3 479 3069
\end{tabular}
\linebreak{}
%\pagebreak{}

\begin{publications}
\hrulefill \\
\printbib{papers}
\printbib{conferences}
\printbib{techreports}
\printbib{curpapers}
\end{publications}

\end{document}
